\subhead{Introduction}

The {\bf Basic} macro package is just that: basic. It is designed
to be compatable with the \tex\ basic package, and as such is
a minimal package for formatting text. The command set it provides
uses the full power of tex, but is very limited, and doesn't
provide macros for handling environments that are very common in
text formatting.

This package does provide the standard math symbols found in the
tex manual. If you don't plan on using these, there is a stripped
package, {\ss sbasic,} available. It provides only the math mode commands
listed in the main body of the manual - those that aren't math
symbols, but symbols useful in general text formatting. If you
use the full basic format, the first free font file is font file
number. In the stripped package, first free font file is number 8.

The commands are:

{\describe

\bold{bf} Change to boldface typeface: {\bf that looks like this.}

\bold{rm} Use the standard roman typeface. This is the default
typeface for this package.

\bold{it} Change to italic typeface: {\it that looks like this.}

\bold{ss} Change to sans serif typeface: {\ss that looks like this.}

\bold{sc} Change to script typeface: {\sc that looks like this.}

\bold{ctrline} This command takes one argument. It centers that
argument on a line by itself. It does not put in extra lines
other than that one line, so your text will look like
\ctrline{this line centered} followed by more text on the next line.
}	% end of basic command set


{\ss sbasic,} available. It provides only the math mode commands
listed in the main body of the manual - those that aren't ma