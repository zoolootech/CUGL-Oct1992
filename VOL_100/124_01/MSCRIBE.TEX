\subhead{Introduction}

The {\ss Scribe} macro package is designed to give the tex user
some of the convenience found in the {\ss scribble} text
formatter. The {\ss Scribe} package consists of a set of commands
borrowed from {\ss scribble}, using the {\ss scribble}
philosophy. This philosophy dictates that the user doesn't want
or need detailed control over what goes on the page. Hence, there
are a set of commands that generate environments similar to
things that people like to do in documents. These environments
come in three flavors: inline, vertical and switches. The inline
environments take an argument, and cause no vertical spacing. The
vertical environments start a new line when they are invoked, and
most of them don't take arguments. The switch environments
generally don't take arguments, and don't generate new lines. All
switch environments change the current typeface.

In addition to these commands, the {\ss scribe} package includes
all the math mode commands listed in the section of the tex users
manual on math mode. This only includes the symbols of general
interest, and not all of the math mode commands normally
available. If you need these commands, use the {\ss scribe+}
package. This contains all of the math mode commands listed in
the appendix, except for the script characters.

\subhead{The Switches}

The switches are commands that change the current typeface. These
are:

{\describe

\bold{r} Change to a roman typeface. This font is the default
typeface for the {\ss scribe} package.

\bold{t} Exactly the same as the {\ss r} command. {\ss t} is an
abbreviation for {\it text}, which is the default environment.

\bold{b} Change to a bold typeface: {\bf that looks like this.}

\bold{i} Change to an italic typeface: {\it that looks like this.}

\bold{s} Change to a sans-serif typeface: {\ss that looks like this.}

\bold{B} Change to a large bold typeface. This face is used for
major headings in the scribe package, and cannot be gotten to
from the package the manual is set in.
}		% end of the switches

\subhead{The Inline Environments}

The inline environments all take an argument. These commands
generally cause a change of font over there argument, without
generating any vertical whitespace in the output document. These
are:

{\describe

\bold{+} The argument to the {\ss +} command is printed
raised above the normal print line, in a position appropriate for
superscripts: for example$\+{this}$ is a super script.

\bold{-} The {\ss -} command is identical to the {\ss +} command,
except that it generates subscripts. For example$\-{this}$ is a subscript.

\bold{ux} The {\ss ux} command underlines every character in its
argument, \underline{like so.}{}

\bold{example} This command prints its arguments in a sans-serif
typeface, imitating (as best as pfont can) a typewriter.
}			% end of the inline environments.

\subhead{The Vertical Environments}

The vertical environments all cause a vertical skip when they are
invoked. In general, they correspond to the pieces of a document
- headings, quotes, verse, addresses, etc.  Some of these
commands expect an argument, and act only on that argument.
These are noted in the following list. Those that do not take
arguments cause an tex environment change, and should be have
a pair of curly braces to designate the text the command applies
to, like so: {\ss \{\\command text for it to apply to\}.}

{\describe

\bold{center} Centers its argument. This command starts a new
line for its argument, and a second new line for the text
following its argument.

\bold{majorhead} This command prints its argument in the {\ss B}
typeface, centered on the page. It leaves 2/3's of an inch above
and 1/6 of an inch plus paragraph spacing below its argument. The
piece of text that follows this command will be indented as a paragraph.

\bold{head} This command prints its argument in the {\ss b}
typeface, left justified. It leaves 1/3 of an inch above and 1/6
of an inch plus paragraph spacing below its argument. The piece
of text that follows this command will be indented a paragraph.

\bold{subhead} This command prints its argument in the {\ss i}
typeface, left justified. It leaves 1/6 of an inch above and
paragraph spacing below its argument. The piece of text that
follows this command will be indented as a paragraph.

\bold{address} Non-filled, left justified environment. This
environment is indented approximately 4 inches from the left
margin. It is suitable for the return address on a letter.

\bold{closing} Identical to the {\ss address} environment, with a
name more appropriate for the closing of a letter.

\bold{verbatim} This environment is non-filled. Other than that,
it is identical to the surrounding environment. It is suitable
for including text fragments that you don't want formatted.

\bold{display} Identical to {\ss verbatim}, except that the font
is changed to the {\ss s} font. Usefull for including program
text, or long examples.

\bold{verse} A non-filled environment indented by 1/2 an inch. As
its name indicates, it is suitable for verse.

\bold{quote} A filled environemt indented by 1/2 an inch.
Suitable for use with long text quotations.

\bold{describe} This is an outdented environment. It is useful
for a list of descriptions. In this environment, the {\ss @}
command is useful for the first object being described. The {\ss
@} command causes a change to a bold typeface, and insures that
at least one space follows its argument.
}			% end of the vertical environments

\subhead{The Rest of the Commands}

The following commands are added to make users of scribble and
scribe more comfortable.

{\describe

\bold{newpage} Skip to the top of the next page.

\bold{blankspac} Skip left by the number of points specified by
its argument.

\bold{blankline} Skip down by the number of points specified by
its argument.

\bold{style} The arguments of the {\ss style} command is passed
to {\FF Fancy Font} as parameters. It is used exactly like the
builtin {\ss ff} command.
}			% end of the miscellanious crap
suitable for verse.

\bold{quote} A filled environemt indented by 1/2 